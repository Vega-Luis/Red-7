%% LyX 2.3.4.2 created this file.  For more info, see http://www.lyx.org/.
%% Do not edit unless you really know what you are doing.
\documentclass[letterpaper,leqno,spanish,english,compsoc]{IEEEtran}
\usepackage[T1]{fontenc}
\usepackage[latin9]{inputenc}
\usepackage{fancyhdr}
\pagestyle{fancy}
\synctex=-1
\usepackage{endnotes}
\usepackage{setspace}

\makeatletter

%%%%%%%%%%%%%%%%%%%%%%%%%%%%%% LyX specific LaTeX commands.
\pdfpageheight\paperheight
\pdfpagewidth\paperwidth


%%%%%%%%%%%%%%%%%%%%%%%%%%%%%% Textclass specific LaTeX commands.
\let\footnote=\endnote
\newlength{\lyxlabelwidth}      % auxiliary length 

\makeatother

\usepackage{babel}
\addto\shorthandsspanish{\spanishdeactivate{~<>}}

\usepackage{listings}
\lstset{language={[ANSI]C++},
float=h,
numbers=left,
numberstyle={\tiny},
stepnumber=1,
basicstyle={\footnotesize\ttfamily},
tabsize=2}
\begin{document}
\selectlanguage{spanish}%
\begin{titlepage}
\begin{center}
\textbf{\large{}Instituto Tecnol�gico de Costa Rica, sede Cartago}\\
\textbf{\large{}Escuela de computaci�n}\\
\textbf{\large{}Programaci�n L�gica}{\large\par}
\par\end{center}

\vspace{1.25cm}

\begin{center}
\textbf{\LARGE{}Proyecto 01}\\
\textbf{\LARGE{}Red 7 }{\LARGE\par}
\par\end{center}

\begin{singlespace}
\begin{center}
\vspace{1cm}
est. Luis �ngel Vega Rodr�guez 2018161651 \\
est. Marcos M�ndez H�dalgo\\
Prof. Ing. Luis Roberto Villalobos Arias 
\par\end{center}
\end{singlespace}

\smallskip{}

\begin{center}
Semestre 2\\
28 de noviembre, 2020
\par\end{center}

\vspace{1cm}

Resumen ejecutivo

\end{titlepage}

\section*{{\LARGE{}Introducci�n }}
La programaci�n l�gica es un paradigma poco conocido pero, sin embargo, poderoso.
Una de las mejores formas de aprender un nuevo lenguaje o paradigma, es por supuesto, la pr�ctica y es a�n m�s eficiente cuando esta es divertida y f�cil de percibir.

Red 7 es un juego de mesa por turnos donde cada jugador inicia con 7 cartas las cuales debe usar de la forma m�s inteligente posible, para 


\subsection*{{\LARGE{}Objetivos }}
Incursionar en la programaci�n l�gica por medio del lenguaje ProLog.
Aplicar los concimientos de l�gica s�mbolica a la programci�n en ProLog.
Experimentar el verdadero poder de la programaci�n l�gica.
\subsection*{{\LARGE{}Delimitaciones }}

\subsection*{{\LARGE{}Alcance }}

\subsection*{{\LARGE{}Definici�n del problema }}

\section*{{\LARGE{}Valoraciones }}

\section*{{\LARGE{}Concluciones}}

\section*{{\LARGE{}Referencias}}\selectlanguage{english}%

\end{document}
