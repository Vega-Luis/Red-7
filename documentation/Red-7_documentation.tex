%% LyX 2.3.4.2 created this file.  For more info, see http://www.lyx.org/.
%% Do not edit unless you really know what you are doing.
\documentclass[letterpaper,leqno,spanish,english,compsoc]{IEEEtran}
\usepackage[T1]{fontenc}
\usepackage[latin9]{inputenc}
\usepackage{fancyhdr}
\usepackage{cite}
\pagestyle{fancy}
\synctex=-1
\usepackage{endnotes}
\usepackage{setspace}

\makeatletter

%%%%%%%%%%%%%%%%%%%%%%%%%%%%%% LyX specific LaTeX commands.
\pdfpageheight\paperheight
\pdfpagewidth\paperwidth


%%%%%%%%%%%%%%%%%%%%%%%%%%%%%% Textclass specific LaTeX commands.
\let\footnote=\endnote
\newlength{\lyxlabelwidth}      % auxiliary length 

\makeatother

\usepackage{babel}
\addto\shorthandsspanish{\spanishdeactivate{~<>}}

\usepackage{listings}
\lstset{language={[ANSI]C++},
float=h,
numbers=left,
numberstyle={\tiny},
stepnumber=1,
basicstyle={\footnotesize\ttfamily},
tabsize=2}
\begin{document}
\selectlanguage{spanish}%
\begin{titlepage}
\begin{center}
\textbf{\large{}Instituto Tecnol�gico de Costa Rica, sede Cartago}\\
\textbf{\large{}Escuela de computaci�n}\\
\textbf{\large{}Programaci�n L�gica}{\large\par}
\par\end{center}

\vspace{1.25cm}

\begin{center}
\textbf{\LARGE{}Proyecto 01}\\
\textbf{\LARGE{}Red 7 }{\LARGE\par}
\par\end{center}

\begin{singlespace}
\begin{center}
\vspace{1cm}
est. Luis �ngel Vega Rodr�guez 2018161651 \\
est. Marcos M�ndez H�dalgo\\
Prof. Ing. Luis Roberto Villalobos Arias 
\par\end{center}
\end{singlespace}

\smallskip{}

\begin{center}
Semestre 2\\
28 de noviembre, 2020
\par\end{center}

\vspace{1cm}

Resumen ejecutivo

\end{titlepage}

\section*{{\LARGE{}Introducci�n }}
La programaci�n l�gica es un paradigma poco conocido pero, sin embargo, poderoso.
Una de las mejores formas de aprender un nuevo lenguaje o paradigma, es por supuesto, la pr�ctica y es a�n m�s eficiente cuando esta es divertida y f�cil de percibir.

Red 7 es un juego de mesa por turnos donde cada jugador inicia con 7 cartas las cuales debe usar de la forma m�s inteligente posible, para 


\subsection*{{\LARGE{}Objetivos }}
Incursionar en la programaci�n l�gica por medio del lenguaje ProLog.
Aplicar los concimientos de l�gica s�mbolica a la programci�n en ProLog.
Experimentar el verdadero poder de la programaci�n l�gica.
\subsection*{{\LARGE{}Delimitaciones }}
Uno de los mayores inconvenientes para empezar con el desarrollo del proyecto, fue la inexperiencia con el lenguaje de programaci�n Prolog, 
no es un lenguaje que se acostumbre ver en profundidad en la carrera. A eso se le suma que nunca se hab�a hecho una IA antes, est� es la primera
vez que se realiza una. Uno de los estudiantes present� problemas mayores con su internet, esto debido a la zona en la que vive, lo cu�l produjo
que se entorpecieran algunos procesos de uni�n de c�digo e ideas.

\subsection*{{\LARGE{}Alcance }}
El proyecto se prevee que no tenga un gran alcance, despu�s de todo es realizado por unos estudiantes con el fin de tener un mejor desarrollo
acad�mico, por lo que todo el trabajo que se haga es con el fin de aprender lo m�s que se pueda y no el de presentar un desarrollo que cumpla
con lo excelente. En ocasiones la ruta a la excelencia no es siempre por la que m�s se aprende.

\subsection*{{\LARGE{}Definici�n del problema }}
El problema de primeras no se ve tan complejo, hay que desarollar una inteligencia artificial que sea capaz de jugar el juego de mesa Red 7 en contra
de una persona y varias inteligencia artificiales. A medida que se desarroll� el proyecto, se notaron una gran cantidad de subproblemas arraigados o bien
al juego o bien a la realizaci�n de una interfaz gr�fica dirigida tanto por un jugador real como por una inteligencia artificial.

\section*{{\LARGE{}Valoraciones }}
Fue necesario valorar en como representar el sistema de cartas en el juego, tambi�n todo lo referente a manejo de jugadores, turnos, derrota, victoria, empates. //
Adem�s se tuvo que pensar en como dise�ar el comportamiento de la inteligencia artificial(Backtracking) para que este no fuera en exceso agresivo ni muy simple,
esta inteligencia debe de representar un reto a lo suficiente como para que la persona que juegue tenga que realmente esforzarce por ganarle a la inteligencia, pero 
tampoco se buscaba realizar algo imposible de derrotar.

\section*{{\LARGE{}Concluciones}}
Prolog es un excelente lenguaje para el desarrollo de IA. //
Una IA no siempre se ve envuelta con machine learning. //
Prolog es un lenguaje que puede parecer rigido, pero si se le da un buen uso termina siendo todo lo contrario.
Python es un excelente lenguaje para realizar ligas con prolog, hay abundancia de recursos para hacerlo.
Sin duda alguna, una IA tiene una gran ventaja sobre una persona gracias a la cantidad de datos que puede procesar en un instante.
\cite{aggregate_prolog_2020} \cite{numlist_prolog_2020} \cite{Mungall_2017} \cite{unknown_2010} \cite{Oakley_2015}
\bibliographystyle{IEEEtran}
\bibliography{references}

\end{document}
